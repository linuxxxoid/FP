\documentclass[12pt]{article}

\usepackage{fullpage}
\usepackage{multicol,multirow}
\usepackage{tabularx}
\usepackage{ulem}
\usepackage[utf8]{inputenc}
\usepackage[russian]{babel}
\usepackage{amsmath}
\usepackage{amssymb}

\usepackage{titlesec}

\titleformat{\section}
  {\normalfont\Large\bfseries}{\thesection.}{0.3em}{}

\titleformat{\subsection}
  {\normalfont\large\bfseries}{\thesubsection.}{0.3em}{}

\titlespacing{\section}{0pt}{*2}{*2}
\titlespacing{\subsection}{0pt}{*1}{*1}
\titlespacing{\subsubsection}{0pt}{*0}{*0}
\usepackage{listings}
\lstloadlanguages{Lisp}
\lstset{extendedchars=false,
	breaklines=true,
	breakatwhitespace=true,
	keepspaces = true,
	tabsize=2
}
\begin{document}


\section*{Отчет по лабораторной работе №\,2 
по курсу \guillemotleft  Функциональное программирование\guillemotright}
\begin{flushright}
Студент группы 8О-307 МАИ \textit{Вельтман Лина}, \textnumero 7 по списку \\
\makebox[7cm]{Контакты: {\tt  kluuo@mail.ru} \hfill} \\
\makebox[7cm]{Работа выполнена: 27.03.2020 \hfill} \\
\ \\
Преподаватель: Иванов Дмитрий Анатольевич, доц. каф. 806 \\
\makebox[7cm]{Отчет сдан: \hfill} \\
\makebox[7cm]{Итоговая оценка: \hfill} \\
\makebox[7cm]{Подпись преподавателя: \hfill} \\

\end{flushright}

\section{Тема работы}
Простейшие функции работы со списками Коммон Лисп.

\section{Цель работы}
Научиться конструировать списки, находить элемент в списке, использовать схему линейной и древовидной рекурсии для обхода и реконструкции плоских списков и деревьев.

\section{Задание (вариант №12)}
Запрограммируйте рекурсивно на языке Коммон Лисп функцию, подсчитывающую число вхождений заданного целого числа в дерево.

\section{Оборудование ПЭВМ студента}
Ноутбук MacBook Pro (13-inch, 2017), процессор 2.3GHz Intel Core i5, память: 8Gb, разрядность системы: 64.

\section{Программное обеспечение ЭВМ студента}
macOS Catalina 10.15.4, реализация языка SBCL 1.4.16, текстовый редактор Sublime Text 3.

\section{Идея, метод, алгоритм}
Функция {\tt count-int} принимает в качестве первого аргумента целое число и в качестве второго аргумента список. Нужно найти количество вхождений первого числа в дерево. Работает она следующим образом:
\begin{enumerate}
\setlength{\itemsep}{-1mm} % уменьшает расстояние между элементами списка
\item При помощи ключевого слова {\tt \&aux} определяем локальные переменные: {\tt head} и {\tt tail}, которые равны первому элементу списка (атом) и оставшемуся списку без первого элемента соответственно. 
\item Для реализации ветвления алгоритма я использую {\tt cond}. В качестве первого условия идет проверка на пустоту головы списка, ведь если головы списка нет, то и самого списка тоже. Если условие выполняется, то выводим 0 на экран и заканчиваем работу.
\item Вторым условием является проверка головы на атомарность, если голова списка - атом, то я сравниваю его с заданным целым числом {\tt number}. Если равенство выполняется, то делаю сложение единицы с рекурсивным вызовом функции {\tt count-int} с аргументами: заданное целое число {\tt number} и хвост списка {\tt tail}, иначе выполняется вызов функции {\tt count-int} c такими же аргументами, но без сложения с единицей.
\item Когда ни одно из вышеописанных условий не выполняется, то в качестве последнего условия совершается сложение двух рекурсивных вызовов функций {\tt count-int}, аргументами первой выступают: заданное целое число {\tt number} и голова списка {\tt head}, второй: заданное целое число {\tt number} и хвост списка {\tt tail}.
\end{enumerate}

\section{Сценарий выполнения работы}

\section{Распечатка программы и её результаты}

\subsection{Исходный код}
\lstinputlisting{lab2_12.lisp}

\subsection{Результаты работы}
\lstinputlisting{log.txt}

\section{Дневник отладки}
\begin{tabular}{|c|c|c|c|}
\hline
Дата & Событие & Действие по исправлению & Примечание \\
\hline
\end{tabular}

\section{Замечания автора по существу работы}
Было интересно поработать с рекурсией в {\tt Common lisp}.

\section{Выводы}
Во время выполнения данной лабораторной я познакомилась с функциями обработки списков: {\tt car}, {\tt cdr}. Реализовала древовидную рекурсию для обхода. Применила {\tt сond}-выражения в своем алгоритме.

\end{document}