\documentclass[12pt]{article}

\usepackage{fullpage}
\usepackage{multicol,multirow}
\usepackage{tabularx}
\usepackage{ulem}
\usepackage[utf8]{inputenc}
\usepackage[russian]{babel}
\usepackage{amsmath}
\usepackage{amssymb}

\usepackage{titlesec}

\titleformat{\section}
  {\normalfont\Large\bfseries}{\thesection.}{0.3em}{}

\titleformat{\subsection}
  {\normalfont\large\bfseries}{\thesubsection.}{0.3em}{}

\titlespacing{\section}{0pt}{*2}{*2}
\titlespacing{\subsection}{0pt}{*1}{*1}
\titlespacing{\subsubsection}{0pt}{*0}{*0}
\usepackage{listings}
\lstloadlanguages{Lisp}
\lstset{extendedchars=false,
	breaklines=true,
	breakatwhitespace=true,
	keepspaces = true,
	tabsize=2
}
\begin{document}


\section*{Отчет по лабораторной работе №\,1 
по курсу \guillemotleft  Функциональное программирование\guillemotright}
\begin{flushright}
Студент группы 8О-307 МАИ \textit{Вельтман Лина}, \textnumero 7 по списку \\
\makebox[7cm]{Контакты: {\tt  kluuo@mail.ru} \hfill} \\
\makebox[7cm]{Работа выполнена: 20.03.2020 \hfill} \\
\ \\
Преподаватель: Иванов Дмитрий Анатольевич, доц. каф. 806 \\
\makebox[7cm]{Отчет сдан: \hfill} \\
\makebox[7cm]{Итоговая оценка: \hfill} \\
\makebox[7cm]{Подпись преподавателя: \hfill} \\

\end{flushright}

\section{Тема работы}
Примитивные функции и особые операторы в языке Common Lisp.

\section{Цель работы}
Научиться вводить S-выражения в Лисп-систему, определять переменные и функции, работать с условными операторами, работать с числами, используя схему линейной и древовидной рекурсии.

\section{Задание (вариант №7)}
Запрограммируйте на языке Коммон Лисп функцию с трёмя параметрами - действительными числами. Функция должна возвращать три значения с помощью values:
\begin{itemize}
\item возвести в квадрат те параметры, значения которых неотрицательно,
\item вернуть сами параметры, которые отрицательны.
\end{itemize}
\section{Оборудование ПЭВМ студента}
Ноутбук MacBook Pro (13-inch, 2017), процессор 2.3GHz Intel Core i5, память: 8Gb, разрядность системы: 64.

\section{Программное обеспечение ЭВМ студента}
macOS Catalina 10.15.4, реализация языка SBCL 1.4.16, текстовый редактор Sublime Text 3.
\section{Идея, метод, алгоритм}
Функция {\tt three-square } принимает в качестве аргументов 3 числа и работает следующим образом:
\begin{enumerate}
\setlength{\itemsep}{-1mm} % уменьшает расстояние между элементами списка
\item По заданию нужно использовать примитив values, который позволяет принимать любое количество аргументов и возвращает столько же значений. В моем случае последняя форма тела функции является values с тремя аргументами, то вызов такой функции как раз и вернёт три значения.
\item Аргументами values является функция-проверка заданных по варианту условий - {\tt checker}, которая принимает одно значение в качестве аргумента.
\item Внутри {\tt checker} происходит проверка полученного значения на нестрогое неравенство с нулем. Если число больше нуля или равно ему, то возводим его в квадрат, если нет, то число остается без изменений.
\item Измененные или оставшиеся в исходном виде числа возвращаются из функции в values и выводятся на экран.
\end{enumerate}

\section{Сценарий выполнения работы}

\section{Распечатка программы и её результаты}

\subsection{Исходный код}
\lstinputlisting{lab1_7.lisp}

\subsection{Результаты работы}
\lstinputlisting{log.txt}

\section{Дневник отладки}
\begin{tabular}{|c|c|c|c|}
\hline
Дата & Событие & Действие по исправлению & Примечание \\
\hline
\end{tabular}

\section{Замечания автора по существу работы}
Работа показалась мне слишком простой с точки зрения программирования. 

\section{Выводы}
Для выполнения данной лабораторной мне пришлось познакомиться с новым для меня языком программирования {\it Common Lisp}. Получила первый опыт: написала простейшую программу с использованием стандартных составляющих языка.\\
Программа работает за константные время и память, поскольку принимает фиксированное количество аргументов и количество действий, выполняемых программой ограничено.\\
Так как моя программа оказалось довольно простой, мне не удалось поработать с рекурсией. Надеюсь, что при выполнении будущих лабораторных я смогу получить более трудное задание, чтобы поглубже изучить особенности пройденного материала.

\end{document}