\documentclass[12pt]{article}

\usepackage{fullpage}
\usepackage{multicol,multirow}
\usepackage{tabularx}
\usepackage{ulem}
\usepackage[utf8]{inputenc}
\usepackage[russian]{babel}
\usepackage{amsmath}
\usepackage{amssymb}

\usepackage{graphicx}
\DeclareGraphicsExtensions{.png}

\usepackage{titlesec}

\titleformat{\section}
  {\normalfont\Large\bfseries}{\thesection.}{0.3em}{}

\titleformat{\subsection}
  {\normalfont\large\bfseries}{\thesubsection.}{0.3em}{}

\titlespacing{\section}{0pt}{*2}{*2}
\titlespacing{\subsection}{0pt}{*1}{*1}
\titlespacing{\subsubsection}{0pt}{*0}{*0}
\usepackage{listings}
\lstloadlanguages{Lisp}
\lstset{extendedchars=false,
	breaklines=true,
	breakatwhitespace=true,
	keepspaces = true,
	tabsize=2
}
\begin{document}


\section*{Отчет по лабораторной работе №\,5 
по курсу \guillemotleft  Функциональное программирование\guillemotright}
\begin{flushright}
Студент группы 8О-307 МАИ \textit{Вельтман Лина}, \textnumero 7 по списку \\
\makebox[7cm]{Контакты: {\tt  kluuo@mail.ru} \hfill} \\
\makebox[7cm]{Работа выполнена: 14.05.2020 \hfill} \\
\ \\
Преподаватель: Иванов Дмитрий Анатольевич, доц. каф. 806 \\
\makebox[7cm]{Отчет сдан: \hfill} \\
\makebox[7cm]{Итоговая оценка: \hfill} \\
\makebox[7cm]{Подпись преподавателя: \hfill} \\

\end{flushright}

\section{Тема работы}
Обобщённые функции, методы и классы объектов.

\section{Цель работы}
Научиться определять простейшие классы, порождать экземпляры классов, считывать и изменять значения слотов, научиться определять обобщённые функции и методы.

\section{Задание (вариант №23)}
Определить обобщённую функцию и методы {\tt on-single-line3-p} - предикат,
принимающий в качестве аргументов три точки (радиус-вектора) и необязательный параметр {\tt tolerance} (допуск),
возвращающий {\tt T}, три указанные точки лежат на одной прямой (вычислять с допустимым отклонением {\tt tolerance}).
Точки могут быть заданы как декартовыми координатами (экземплярами {\tt cart}), так и полярными (экземплярами {\tt polar}).

\begin{lstlisting}[language=Lisp]
(defgeneric on-single-line3-p (v1 v2 v2 &optional tolerance))
(defmethod on-single-line3-p ((v1 cart) (v2 cart) (v3 cart) &optional (tolerance 0.001))
  ...)
\end{lstlisting}

\section{Оборудование ПЭВМ студента}
Ноутбук MacBook Pro (13-inch, 2017), процессор 2.3GHz Intel Core i5, память: 8Gb, разрядность системы: 64.

\section{Программное обеспечение ЭВМ студента}
macOS Catalina 10.15.4, реализация языка SBCL 1.4.16, текстовый редактор Sublime Text 3.

\section{Идея, метод, алгоритм}
Воспользуемся уравнением прямой, проходящей через 2 точки. Если 3 точки лежат на одной прямой, то для них выполняется равенство:
$$\frac{x_3 - x_1}{x_2 - x_1} = \frac{y_3 - y_1}{y_2 -y_1}.$$
Чтобы избежать погрешностей при делении, заменим частные на произведения:
$$(x_2 - x_1)*(y_3 - y_1) = (x_3 - x_1)*(y_2 - y_1).$$
Функция {\tt on-single-line3-p} проверяет выполнение вышеприведенного равенства для всех элементов списка с помощью функции {\tt on-single-line3}. Функция умеет работать как с декартовыми координатами, так и с полярными.

\section{Сценарий выполнения работы}

\section{Распечатка программы и её результаты}

\subsection{Исходный код}
\lstinputlisting{lab5_23.lisp}

\subsection{Результаты работы}
\lstinputlisting{log.txt}

\section{Дневник отладки}
\begin{tabular}{|l|p{180pt}|p{180pt}|}
\hline
Дата & Событие & Действие по исправлению \\ \hline
19.05.2020 & Некорректная работа с полярными координатами & Добавлена функция по обработке полярных координат\\
\hline
\end{tabular}

\section{Замечания, выводы}
Благодаря данной лабораторной работы я познакомилась с классами и обобщенными функциями в языке Common Lisp. Простой и понятный синтаксис для создания классов является преимуществом. Также хотелось бы упомянуть, что обощенные функции позволяют не создавать несколько похожих функций для разных классов, а просто написать реализации одной функции, которая будет применима для нескольких классов, что очень удобно.

\end{document}